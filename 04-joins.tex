\documentclass{beamer}
% 
% Choose how your presentation looks.
% 
% For more themes, color themes and font themes, see:
% http://deic.uab.es/~iblanes/beamer_gallery/index_by_theme.html
% 
\mode<presentation>{
  %\usetheme{Madrid}      % or try Darmstadt, Madrid, Warsaw, ...
  \usetheme{Boadilla}
  \usecolortheme{beaver} % or try albatross, beaver, crane, ...
  \usefonttheme{default}  % or try serif, structurebold, ...
  \setbeamertemplate{navigation symbols}{}
  \setbeamertemplate{caption}[numbered]
} 

\usepackage{graphicx}
\usepackage[english]{babel}
\usepackage[utf8x]{inputenc}
\usepackage{tabulary}

%% For Pretty In-Line Code:
\usepackage{listings}
\usepackage{color}

\definecolor{dkgreen}{rgb}{0,0.6,0}
\definecolor{gray}{rgb}{0.95,0.95,0.95}
\definecolor{mauve}{rgb}{0.58,0,0.82}

\lstdefinestyle{customsql}{
  xleftmargin=4ex
  ,language=SQL
}


\lstset{
  frame=none,
  language=SQL,
  aboveskip=3mm,
  belowskip=3mm,
  showstringspaces=false,
  columns=flexible,
  basicstyle={\small\ttfamily},
  numbers=left,
  numberstyle=\tiny\color{black},
  keywordstyle=\color{blue},
  commentstyle=\color{dkgreen},
  backgroundcolor=\color{gray},
  stringstyle=\color{mauve},
  breaklines=true,
  breakatwhitespace=true,
  tabsize=2,
  captionpos=b,
  escapechar=@,
  style=customsql
}

\AtBeginSection[]{
  \begin{frame}<beamer>
    \frametitle{Outline for section \thesection}
    \tableofcontents[currentsection]
  \end{frame}
}

\title[SQL Survival Guide]{SQL Survival Guide:\\Joins}
\author{Andy Choens}
\institute{EBCoP}
\date{June 17, 2015}

\begin{document}

\begin{frame} %%%%%%%%%%%%%%%%%%%%%%%%%%%%%%%%%%%%%%%%%%%%%%%%%%%%%%%%%%%%%%%%%%
  \titlepage
\end{frame}

\begin{frame}{Outline} %%%%%%%%%%%%%%%%%%%%%%%%%%%%%%%%%%%%%%%%%%%%%%%%%%%%%%%%%
  \tableofcontents
\end{frame}




\section{Resources} %%%%%%%%%%%%%%%%%%%%%%%%%%%%%%%%%%%%%%%%%%%%%%%%%%%%%%%%%%%

\subsection{Learning Resources} %%%%%%%%%%%%%%%%%%%%%%%%%%%%%%%%%%%%%%%%%%%%%%%%

\begin{frame} %%%%%%%%%%%%%%%%%%%%%%%%%%%%%%%%%%%%%%%%%%%%%%%%%%%%%%%%%%%%%%%%%%
  \frametitle{SQL Survival Guide}

  \begin{itemize}
  \item \textbf{Additional Presentation Resources:}
    \begin{itemize}
    \item \url{https://github.com/choens/sql-survival-guide}
    \item \textbf{SQL Queries:} ``sql/04-joins/''
    \item \textbf{Data Provided (SQLite):} ``data/04-joins.sqlite''
    \end{itemize}
  \item \textbf{Further Reading:}
    \begin{itemize}
    \item \url{https://en.wikipedia.org/wiki/Join\_(SQL)}
    \end{itemize}
  \item Example data in SQLite: data/04-joins.sqlite.
  \end{itemize}
  
\end{frame}

\subsection{Example Data} %%%%%%%%%%%%%%%%%%%%%%%%%%%%%%%%%%%%%%%%%%%%%%%%%%%%%%

\begin{frame} %%%%%%%%%%%%%%%%%%%%%%%%%%%%%%%%%%%%%%%%%%%%%%%%%%%%%%%%%%%%%%%%%%
  \frametitle{Example Data}

  \begin{itemize}
  \item \textbf{Two tables of example data:}
    \begin{itemize}
    \item DEPARTMENTS
    \item EMPLOYEES
    \end{itemize}
  \item \textbf{SQL Queries:} ``sql/04-joins/''
    \begin{itemize}
    \item create\_tables.sql: Creates the example tables.
    \item Other files contain example queries.
    \end{itemize}
  \end{itemize}

\end{frame}

\begin{frame} %%%%%%%%%%%%%%%%%%%%%%%%%%%%%%%%%%%%%%%%%%%%%%%%%%%%%%%%%%%%%%%%%%

  \frametitle{Table: DEPARTMENTS}

  \begin{center}
    \begin{tabulary}{\textwidth}{CLC}
      DEPT\_ID & DEPT\_NAME  & DEPT\_FLOOR \\
      \hline
      31       & Sales       & 1           \\
      33       & Engineering & 3           \\
      34       & Clerical    & 2           \\
      35       & Marketing   & 3           \\
    \end{tabulary}

    \bigskip
    \textbf{Aliased as `dept'.}
  \end{center}

\end{frame}

\begin{frame} %%%%%%%%%%%%%%%%%%%%%%%%%%%%%%%%%%%%%%%%%%%%%%%%%%%%%%%%%%%%%%%%%%
  \frametitle{Table: EMPLOYEES}

  \begin{center}
    \begin{tabulary}{\textwidth}{RCLLC}
      EID & DEPT\_ID & LAST\_NAME & FIRST\_NAME & GENDER\\
      \hline
      1   & 31       & Rafferty   & Gerry       & M     \\
      3   & 33       & Jones      & Jon         & M     \\
      5   & 33       & Heisenberg & Werner      & M     \\
      7   & 34       & Robinson   & Elizabeth   & F     \\
      9   & 34       & Smith      & Jefferson   & M     \\        
      11  & NULL     & Williams   & Serena      & F     \\
    \end{tabulary}

    \bigskip
    \textbf{Aliased as `empl'.}
  \end{center}
  
\end{frame}



\section{Dafynitions} %%%%%%%%%%%%%%%%%%%%%%%%%%%%%%%%%%%%%%%%%%%%%%%%%%%%%%%%%%

\begin{frame} %%%%%%%%%%%%%%%%%%%%%%%%%%%%%%%%%%%%%%%%%%%%%%%%%%%%%%%%%%%%%%%%%%
  \frametitle{Dafynitions}

  \textbf{Why:}
  \begin{itemize}
  \item We need a common set of terms to discuss SQL Joins.
  \item If I use a term you don't recognize, stop me.
  \end{itemize}

\end{frame}  

\begin{frame} %%%%%%%%%%%%%%%%%%%%%%%%%%%%%%%%%%%%%%%%%%%%%%%%%%%%%%%%%%%%%%%%%%
  \frametitle{Dafynition: Relational Data \footnote{https://en.wikipedia.org/wiki/Relational\_database}}

  \textbf{Relational Data:}
  \begin{itemize}
  \item Based on the relational model, as proposed by E.F. Codd in
    1970.
  \item Data is organized into one or more tables (relations) with a
    unique key for each row.
  \item Foreign keys make it possible to link rows in one table to
    rows in another table.
    \begin{itemize}
    \item In the EMPLOYEES table, DEPT\_ID is a foreign key.
    \end{itemize}
  \end{itemize}

\end{frame}  

\begin{frame} %%%%%%%%%%%%%%%%%%%%%%%%%%%%%%%%%%%%%%%%%%%%%%%%%%%%%%%%%%%%%%%%%%
  \frametitle{Dafynition: Clause\footnote{https://en.wikipedia.org/wiki/SQL\#Language\_elements}}
  
  The following protected words denote a SQL clause:

  \smallskip
  \begin{itemize}
  \item SELECT
  \item FROM
  \item WHERE
  \item HAVING
  \end{itemize}
  
  \bigskip
  \pause
  Style:
  \begin{itemize}
  \item Align SQL Clauses
  \item Sub-clauses should be consistently indented.
  \end{itemize}
  
\end{frame}  

\begin{frame}[fragile] %%%%%%%%%%%%%%%%%%%%%%%%%%%%%%%%%%%%%%%%%%%%%%%%%%%%%%%%%
  \frametitle{Dafynition: Join Clause\footnote{https://en.wikipedia.org/wiki/Join\_(SQL)}}

  \begin{lstlisting}
    from EMPLOYEES empl inner join DEPARTMENTS dept
    on empl.dept_id = dept.dept_id
  \end{lstlisting}

  \begin{itemize}
  \item The FIRST thing you should write.
  \item Combines records from two or more tables (result set).
  \item SQL Joins are a difficult skill to master.
  \end{itemize}

\end{frame}  



\section{Order Of Operations} %%%%%%%%%%%%%%%%%%%%%%%%%%%%%%%%%%%%%%%%%%%%%%

\begin{frame} %%%%%%%%%%%%%%%%%%%%%%%%%%%%%%%%%%%%%%%%%%%%%%%%%%%%%%%%%%%%%%%%%%
  \frametitle{Order of Operations: Math}

  \textbf{Question: }What is the answer to this equation?

  \bigskip
  \[\sqrt{(2^2+2) \cdot 6}\]

  \pause
  \bigskip
  \textbf{Anwer:} 6

  \bigskip
  \pause
  \textbf{Question: }How were we all able to come up with the same
  answer?

  \pause
  \bigskip
  \textbf{Answer:} Mathematical Order of Operations

\end{frame}

\begin{frame} %%%%%%%%%%%%%%%%%%%%%%%%%%%%%%%%%%%%%%%%%%%%%%%%%%%%%%%%%%%%%%%%%%
  \frametitle{Order of Operations: SQL}

  \textbf{SQL Order Of Operations: }

  \medskip
  \begin{enumerate}
  \item FROM
  \item WHERE
  \item GROUP BY
  \item SELECT
  \item HAVING
  \item ORDER BY
  \end{enumerate}

\end{frame}

\begin{frame} %%%%%%%%%%%%%%%%%%%%%%%%%%%%%%%%%%%%%%%%%%%%%%%%%%%%%%%%%%%%%%%%%%
  \frametitle{Order of Operations: SQL}

  \textbf{SQL Order Of Operations --- Notes:}

  \bigskip
  \begin{itemize}
  \item Sub-queries are run before the outer query.
  \item The optimizer may reorganize query (relational algebra).
  \item Sometimes writing things out of order can bite you.
  \item Examples of such a problem is forthcoming.
  \end{itemize}
  
\end{frame}  

\begin{frame} %%%%%%%%%%%%%%%%%%%%%%%%%%%%%%%%%%%%%%%%%%%%%%%%%%%%%%%%%%%%%%%%%%
  \frametitle{Result Set}

  \begin{itemize}
  \item The FIRST thing you should write.
  \item Combines records from two or more tables (result set).
  \item SQL Joins are a difficult skill to master.
  \item They are necessary for working with normalized, relational
    data.
  \end{itemize}

\end{frame}  



\section{SQL Order Of Operations} %%%%%%%%%%%%%%%%%%%%%%%%%%%%%%%%%%%%%%%%%%%%%%

\begin{frame} %%%%%%%%%%%%%%%%%%%%%%%%%%%%%%%%%%%%%%%%%%%%%%%%%%%%%%%%%%%%%%%%%%
  \frametitle{Order of Operations}

  What is the answer to this (silly) equation?

  \bigskip
  \[\sqrt{(2^2+2) \cdot 6}\]

  \bigskip
  \pause
  And how were we all able to come up with the same answer?

\end{frame}

\begin{frame} %%%%%%%%%%%%%%%%%%%%%%%%%%%%%%%%%%%%%%%%%%%%%%%%%%%%%%%%%%%%%%%%%%
  \frametitle{Order of Operations}

  Just like math, SQL has an order of operations: 

  \medskip
  \begin{enumerate}
  \item FROM
  \item WHERE
  \item GROUP BY
  \item SELECT
  \item HAVING
  \item ORDER BY
  \end{enumerate}

\end{frame}

\begin{frame} %%%%%%%%%%%%%%%%%%%%%%%%%%%%%%%%%%%%%%%%%%%%%%%%%%%%%%%%%%%%%%%%%%
  \frametitle{Order of Operations}

  Some last order of operation notes:

  \bigskip
  \begin{itemize}
  \item Sub-queries are run before the outer query.
  \item The optimizer may reorganize query (relational algebra).
  \item Sometimes writing things out of order can bite you.
  \end{itemize}
  
\end{frame}

\begin{frame} %%%%%%%%%%%%%%%%%%%%%%%%%%%%%%%%%%%%%%%%%%%%%%%%%%%%%%%%%%%%%%%%%%
  \frametitle{Dafynition: Result Set\footnote{https://en.wikipedia.org/wiki/Result\_set}}

  SQL Output (Result Set):

  \begin{itemize}
  \item A set of rows from a database.
  \item Effectively a table.
  \end{itemize}

  \pause
  \begin{center}
  \bigskip \textbf{Pro Tip: } Imagine that each step in the order of
  operations produces a result set, and passes it to the next next
  clause.

  \end{center}


\end{frame}  

\section{ANSI  Joins} %%%%%%%%%%%%%%%%%%%%%%%%%%%%%%%%%%%%%%%%%%%%%%%%%%%%%%%%%%

\begin{frame} %%%%%%%%%%%%%%%%%%%%%%%%%%%%%%%%%%%%%%%%%%%%%%%%%%%%%%%%%%%%%%%%%%
  \frametitle{Types of Joins}

  \textbf{ANSI joins:}

  \smallskip

  \begin{columns}

    \begin{column}{.5\textwidth}
      \begin{enumerate}
      \item {\color{red}Cross Join}
      \item \textbf{Inner Join}
      \item \textbf{Left Outer Join}
      \item Right Outer Join
      \item Full Outer Join
      \end{enumerate}
    \end{column}
    
    \begin{column}{.5\textwidth}
      \begin{enumerate}
      \item {\color{red}DO NOT USE\@!}
      \item \textbf{Very Useful.}
      \item \textbf{Very Useful.}
      \item Most of you should not use this.
      \item You will rarely use this.
      \end{enumerate}
    \end{column}

  \end{columns}

  \bigskip
  \bigskip
  \bigskip
  \begin{itemize}
  \item Some of these are more useful than others.
  \end{itemize}
  

\end{frame}  

\begin{frame}
  \frametitle{Result Set Order}
  
  \textbf{The No Particular Order Guarantee:}

  \begin{itemize}
  \item Example queries and result sets provided for each join type.
  \item Result sets shown ordered by EID or DEPT\_ID for readability.
  \item ANSI SQL does not guarantee the order of the result set,
    unless specified by an Order By clause.
  \end{itemize}

\end{frame}

\subsection{Cross Join} %%%%%%%%%%%%%%%%%%%%%%%%%%%%%%%%%%%%%%%%%%%%%%%%%%%%%%%

\begin{frame} %%%%%%%%%%%%%%%%%%%%%%%%%%%%%%%%%%%%%%%%%%%%%%%%%%%%%%%%%%%%%%%%%%
  \frametitle{Cross Join: Discussion}

  \textbf{Cross Join:}

  \bigskip
  \begin{itemize}
  \item Returns the Cartesian product of the tables in the FROM
    clause.
  \item Incorrectly referred to as a Cartesian Join. 
  \item Can be written explicitly and implicitly.
  \item Writing a Cross Join is {\color{red} \textbf{ALMOST ALWAYS}} a bad idea.
  \end{itemize}
\end{frame}

\begin{frame}[fragile] %%%%%%%%%%%%%%%%%%%%%%%%%%%%%%%%%%%%%%%%%%%%%%%%%%%%%%%%%
  \frametitle{Explicit Cross Join: Example}

  \textbf{Question:} How many rows will the following query return?
  \bigskip

  \begin{lstlisting}[title={\tiny Source: https://github.com/Choens/sql-survival-guide/blob/master/sql/04-joins/cross-join.sql}]
    select *
    from DEPARTMENTS dept cross join EMPLOYEES empl
    ;
  \end{lstlisting}
  
  \bigskip
  \bigskip
  \pause 
  \textbf{Answer:} 24 rows!
  
\end{frame}

\begin{frame} %%%%%%%%%%%%%%%%%%%%%%%%%%%%%%%%%%%%%%%%%%%%%%%%%%%%%%%%%%%%%%%%%%
  \frametitle{Cross Join: Example Result Set}

  \begin{columns}[T]
    \begin{column}{.5\textwidth}
      \textbf{Left Table: DEPARTMENTS}
      
      \medskip
      \begin{center}
        \tiny{
          \begin{tabulary}{\textwidth}{CLCC}
            DEPT\_ID & DEPT\_NAME  & DEPT\_FLOOR &            \\
            \hline
            31       & Sales       & 1           & \rightarrow \\
            31       & Sales       & 1           & \rightarrow \\
            31       & Sales       & 1           & \rightarrow \\
            31       & Sales       & 1           & \rightarrow \\
            31       & Sales       & 1           & \rightarrow \\
            31       & Sales       & 1           & \rightarrow \\
          \end{tabulary}
        }

      \end{center}
    \end{column}
    
    \begin{column}{.5\textwidth}
      \textbf{Right Table: EMPLOYEES}
      
      \medskip
      \begin{center}
        \tiny{
          \begin{tabulary}{\textwidth}{CRCLLC}
           & EID & DEPT\_ID & LAST\_NAME & FIRST\_NAME \\
           \hline
         \leftarrow  & 1   & 31       & Rafferty   & Gerry       \\
         \leftarrow  & 3   & 33       & Jones      & Jon         \\
         \leftarrow  & 5   & 33       & Heisenberg & Werner      \\
         \leftarrow  & 7   & 34       & Robinson   & Elizabeth   \\
         \leftarrow  & 9   & 34       & Smith      & Jefferson   \\        
         \leftarrow  & 11  & NULL     & Williams   & Serena      \\
          \end{tabulary}
        }
      \end{center}

    \end{column}

  \end{columns}

  \bigskip
  \begin{center}
    \textbf{N Rows Returned: } (N Rows Left Table) * (N Rows Right Table)
    \[ 4 \cdot 6 = 24 \]
  \end{center}

\end{frame}

\begin{frame} %%%%%%%%%%%%%%%%%%%%%%%%%%%%%%%%%%%%%%%%%%%%%%%%%%%%%%%%%%%%%%%%%%
  \frametitle{Cross Join: Result Set}

  \begin{center}
    {\tiny
      \begin{tabulary}{\textwidth}{RCLLCRLC}
        EID & DEPT\_ID & LAST\_NAME & FIRST\_NAME & GENDER & DEPT\_ID & DEPT\_NAME  & DEPT\_FLOOR \\
        \hline
        1   & 31       & Rafferty   & Gerry       & M      & 31       & Sales       & 1           \\
        1   & 31       & Rafferty   & Gerry       & M      & 33       & Engineering & 3           \\
        1   & 31       & Rafferty   & Gerry       & M      & 34       & Clerical    & 2           \\
        1   & 31       & Rafferty   & Gerry       & M      & 35       & Marketing   & 3           \\
        3   & 33       & Jones      & Jon         & M      & 31       & Sales       & 1           \\
        3   & 33       & Jones      & Jon         & M      & 33       & Engineering & 3           \\
        3   & 33       & Jones      & Jon         & M      & 34       & Clerical    & 2           \\
        3   & 33       & Jones      & Jon         & M      & 35       & Marketing   & 3           \\
        5   & 33       & Heisenberg & Werner      & M      & 31       & Sales       & 1           \\
        5   & 33       & Heisenberg & Werner      & M      & 33       & Engineering & 3           \\
        5   & 33       & Heisenberg & Werner      & M      & 34       & Clerical    & 2           \\
        5   & 33       & Heisenberg & Werner      & M      & 35       & Marketing   & 3           \\
        7   & 34       & Robinson   & Elizabeth   & F      & 31       & Sales       & 1           \\
        7   & 34       & Robinson   & Elizabeth   & F      & 33       & Engineering & 3           \\
        7   & 34       & Robinson   & Elizabeth   & F      & 34       & Clerical    & 2           \\
        7   & 34       & Robinson   & Elizabeth   & F      & 35       & Marketing   & 3           \\
        9   & 34       & Smith      & Jefferson   & M      & 31       & Sales       & 1           \\
        9   & 34       & Smith      & Jefferson   & M      & 33       & Engineering & 3           \\
        9   & 34       & Smith      & Jefferson   & M      & 34       & Clerical    & 2           \\
        9   & 34       & Smith      & Jefferson   & M      & 35       & Marketing   & 3           \\
        11  & [NULL]   & Williams   & Serena      & F      & 31       & Sales       & 1           \\
        11  & [NULL]   & Williams   & Serena      & F      & 33       & Engineering & 3           \\
        11  & [NULL]   & Williams   & Serena      & F      & 34       & Clerical    & 2           \\
        11  & [NULL]   & Williams   & Serena      & F      & 35       & Marketing   & 3           \\
      \end{tabulary}
    }
  \end{center}

\end{frame}

\begin{frame}[containsverbatim] %%%%%%%%%%%%%%%%%%%%%%%%%%%%%%%%%%%%%%%%%%%%%%%%
  \frametitle{Implicit Cross Join: Example}
  
  \textbf{Question:} How many rows will the following query return?
  \bigskip

  \begin{lstlisting}[title={\tiny Source: https://github.com/Choens/sql-survival-guide/blob/master/sql/04-joins/cross-join.sql}]
    select *
    from EMPLOYEES empl, DEPARTMENTS dept
    ;
  \end{lstlisting}

  \bigskip
  \pause
  \textbf{Answer: } Returns the same 24 rows.
  
\end{frame}

\begin{frame} %%%%%%%%%%%%%%%%%%%%%%%%%%%%%%%%%%%%%%%%%%%%%%%%%%%%%%%%%%%%%%%%%%
  \frametitle{Cross Join: Results}
  \textbf{Cross Join:}
  
  \bigskip
  \begin{itemize}
  \item Cross Join == Danger!
  \item Returns the maximum number of rows possible.
  \item I have never written a Cross Join outside of a training environment.
  \item You probably won't either.
  \end{itemize}
\end{frame}

\subsection{Inner Join} %%%%%%%%%%%%%%%%%%%%%%%%%%%%%%%%%%%%%%%%%%%%%%%%%%%%%%%

\begin{frame}[containsverbatim] %%%%%%%%%%%%%%%%%%%%%%%%%%%%%%%%%%%%%%%%%%%%%%%%
  \frametitle{Inner Join: Discussion}

  \textbf{Inner Join:}

  \bigskip
  \begin{itemize}
  \item Returns all records which have matching records in both
    tables, according to the Join clause or WHERE clause.
  \item Can be written explicitly or implicitly.
  \item Returns the least number of rows.
  \end{itemize}
\end{frame}

\begin{frame}[fragile] %%%%%%%%%%%%%%%%%%%%%%%%%%%%%%%%%%%%%%%%%%%%%%%%%%%%%%%%%
  \frametitle{Explicit Inner Join: Example}

  \textbf{Question:} How many rows will the following query return?
  \bigskip

  \begin{lstlisting}[title={\tiny Source: https://github.com/Choens/sql-survival-guide/blob/master/sql/04-joins/cross-join.sql}]
    select *
    from EMPLOYEES empl inner join DEPARTMENTS dept
    on empl.dept_id = dept.dept_id
    ;
  \end{lstlisting}

  \bigskip
  \pause
  \textbf{Answer:} 5 rows

\end{frame}

\begin{frame}[fragile] %%%%%%%%%%%%%%%%%%%%%%%%%%%%%%%%%%%%%%%%%%%%%%%%%%%%%%%%%%%%%%%%%%
  \frametitle{Inner Join: Example Result Set (1)}

  \bigskip
  \begin{columns}[T]
    \begin{column}{.5\textwidth}
      \textbf{Left Table: EMPLOYEES}
      
      \medskip
      \begin{center}
        \tiny{
          \begin{tabulary}{\textwidth}{RCLLC}
            EID & DEPT\_ID & LAST\_NAME & FIRST\_NAME &            \\
            \hline
            1   & 31       & Rafferty   & Gerry       & \rightarrow \\
          \end{tabulary}
        }

      \end{center}
    \end{column}
    
    \begin{column}{.5\textwidth}
      \textbf{Right Table: DEPARTMENTS}
      
      \medskip
      \begin{center}
        \tiny{
          \begin{tabulary}{\textwidth}{CCLC}
                      & DEPT\_ID & DEPT\_NAME  & DEPT\_FLOOR \\
           \hline
           \leftarrow & 31       & Sales       & 1           \\
          \end{tabulary}
        }
      \end{center}

    \end{column}

  \end{columns}

  \bigskip
  \textbf{Result Set: }
  
  \begin{itemize}
  \item Includes all rows in both tables which match via the `on'
    statement.
  \end{itemize}

\end{frame}

\begin{frame}[fragile] %%%%%%%%%%%%%%%%%%%%%%%%%%%%%%%%%%%%%%%%%%%%%%%%%%%%%%%%%%%%%%%%%%
  \frametitle{Inner Join: Example Result Set (2)}

  \bigskip
  \begin{columns}[T]
    \begin{column}{.5\textwidth}
      \textbf{Left Table: EMPLOYEES}
      
      \medskip
      \begin{center}
        \tiny{
          \begin{tabulary}{\textwidth}{RCLLC}
            EID & DEPT\_ID & LAST\_NAME & FIRST\_NAME &            \\
            \hline
            3   & 33       & Jones      & Jon         & \rightarrow \\
            5   & 33       & Heisenberg & Werner      & \rightarrow \\
          \end{tabulary}
        }
      \end{center}
    \end{column}
    
    \begin{column}{.5\textwidth}
      \textbf{Right Table: DEPARTMENTS}
      
      \medskip
      \begin{center}
        \tiny{
          \begin{tabulary}{\textwidth}{CCLC}
                       & DEPT\_ID & DEPT\_NAME  & DEPT\_FLOOR \\
           \hline
           \leftarrow  & 33       & Engineering & 1           \\
           \leftarrow  & 33       & Engineering & 1           \\
          \end{tabulary}
        }
      \end{center}

    \end{column}
  \end{columns}

  \bigskip
  \textbf{Result Set: }
  
  \begin{itemize}
  \item Dept 33 returned twice, because it matches two employees.
  \end{itemize}

\end{frame}

\begin{frame} %%%%%%%%%%%%%%%%%%%%%%%%%%%%%%%%%%%%%%%%%%%%%%%%%%%%%%%%%%%%%%%%%
  \frametitle{Inner Join: Result Set}
  \begin{center}
    {\tiny
      \begin{tabulary}{\textwidth}{RCLLCCLC}
        EID & DEPT\_ID & LAST\_NAME & FIRST\_NAME & GENDER & DEPT\_ID & DEPT\_NAME  & DEPT\_FLOOR \\
        \hline
        1   & 31       & Rafferty   & Gerry       & M      & 31       & Sales       & 1           \\
        3   & 33       & Jones      & Jon         & M      & 33       & Engineering & 3           \\
        5   & 33       & Heisenberg & Werner      & M      & 33       & Engineering & 3           \\
        7   & 34       & Robinson   & Elizabeth   & F      & 34       & Clerical    & 2           \\
        9   & 34       & Smith      & Jefferson   & M      & 34       & Clerical    & 2           \\
      \end{tabulary}
    }
  \end{center}

  \bigskip
  \textbf{Question: } What happened to Dept 35?
  
  \pause
  \bigksip
  \textbf{Answer: } Dropped because there aren't any employees in
  Marketing.

\end{frame}

\begin{frame}[fragile] %%%%%%%%%%%%%%%%%%%%%%%%%%%%%%%%%%%%%%%%%%%%%%%%%%%%%%%%%
  \frametitle{Implicit Inner Join: Example}

  This should look familiar:
  \bigskip

  \begin{lstlisting}[title={\tiny Source: https://github.com/Choens/sql-survival-guide/blob/master/sql/04-joins/cross-join.sql}]
    select *
    from EMPLOYEES empl, DEPARTMENTS dept
    where empl.dept_id = dept.dept_id
    ;
  \end{lstlisting}

  \bigskip
  \textbf{Result Set:}
  \begin{itemize}
  \item Also returns 5 rows.
  \end{itemize}

\end{frame}

\begin{frame} %%%%%%%%%%%%%%%%%%%%%%%%%%%%%%%%%%%%%%%%%%%%%%%%%%%%%%%%%%%%%%%%%%
  \frametitle{Why Not Use Implicit Join Syntax?}

  \textbf{Question:} What's the difference?
  \begin{itemize}
  \item Implicit Cross Join
  \item Implicit Inner Join
  \end{itemize}
  
  \bigskip
  \pause
  \textbf{Answer:} The WHERE clause.
  
  \bigskip
  \begin{itemize}
  \item An Implicit Inner Join is just an optimized Cross Join!
  \item SQL optimizers convert it to explicit Inner Join.
  \item If followed explicitly it could be S.\@L.\@O.\@W.\@
  \item Implicit Join syntax is deprecated. (Geek-speak for don't do
    it.)
  \item Too easy to write a Cross Join.
  \end{itemize}

  
\end{frame}

\begin{frame}
  \frametitle{Other Inner Joins}

  \textbf{We have to move on but:}
  \begin{itemize}
  \item Inner Joins are really useful.
  \item See inner-join.sql for more examples and ways to write Inner Joins.
  \end{itemize}

\end{frame}

\subsection{Left Outer Join} %%%%%%%%%%%%%%%%%%%%%%%%%%%%%%%%%%%%%%%%%%%%%%%%%%

\begin{frame} %%%%%%%%%%%%%%%%%%%%%%%%%%%%%%%%%%%%%%%%%%%%%%%%%%%%%%%%%%%%%%%%%%
  \frametitle{Outer Join}

  \textbf{Outer Join:}

  \begin{itemize}
  \item The result set from Outer Join retains more records than Inner
    Join.
  \item Types of Outer Join:
    \begin{itemize}
    \item Left Outer Join (Left Join)
    \item Right Outer Join (Right Join)
    \item Full Outer Join
    \end{itemize}
  \item Each has different rules for which rows are part of the result set.
  \end{itemize}
\end{frame}

\begin{frame} %%%%%%%%%%%%%%%%%%%%%%%%%%%%%%%%%%%%%%%%%%%%%%%%%%%%%%%%%%%%%%%%%%
  \frametitle{Left Outer Join: Discussion}

  \textbf{Left Outer Join:}

  \begin{itemize}
  \item Result set includes all members of the `left' table.
  \item When a row in the left table does not match any row in the
    right table, there will be NULLS in all columns from the right
    table.
  \item Left Join is the same as Left Outer Join
  \end{itemize}
\end{frame}

\begin{frame}[fragile] %%%%%%%%%%%%%%%%%%%%%%%%%%%%%%%%%%%%%%%%%%%%%%%%%%%%%%%%%
  \frametitle{Left Outer Join: Example} 
  
  \textbf{Question:} How many rows will the following query return?
  \bigskip

  \begin{lstlisting}[title={\tiny Source: https://github.com/Choens/sql-survival-guide/blob/master/sql/04-joins/left-join.sql}]
    select *
    from EMPLOYEES empl left join DEPARTMENTS dept
    on empl.department_id = dept_dept_id
    ;
  \end{lstlisting}
  
  \bigskip
  \textbf{Note: }
  \begin{itemize}
  \item EMPLOYEES is the Left Table.
  \item DEPARTMENTS is the Right Table.
  \item I really hope you can see why.
  \end{itemize}
  
\end{frame}

\begin{frame}[fragile] %%%%%%%%%%%%%%%%%%%%%%%%%%%%%%%%%%%%%%%%%%%%%%%%%%%%%%%%%
  \frametitle{Left Outer Join: Example} 
  
  \textbf{Question:} How many rows will the following query return?
  \bigskip

  \begin{lstlisting}[title={\tiny Source: https://github.com/Choens/sql-survival-guide/blob/master/sql/04-joins/left-join.sql}]
    select *
    from EMPLOYEES empl left join DEPARTMENTS dept
    on empl.department_id = dept_dept_id
    ;
  \end{lstlisting}
  
  \bigskip
  \textbf{Answer:} 6 rows

\end{frame}

\begin{frame} %%%%%%%%%%%%%%%%%%%%%%%%%%%%%%%%%%%%%%%%%%%%%%%%%%%%%%%%%%%%%%%%%%
  \frametitle{Left Outer Join: Example Result Set (1)}

  \begin{columns}[T]
    \begin{column}{.5\textwidth}
      \textbf{Left Table: EMPLOYEES}
      
      \medskip
      \begin{center}
        \tiny{
          \begin{tabulary}{\textwidth}{CCCCC}
            EID & DEPT\_ID & LAST\_NAME & FIRST\_NAME &            \\
            \hline
            1   & 31       & Rafferty   & Gerry       & \rightarrow \\
          \end{tabulary}
        }

      \end{center}
    \end{column}
    
    \begin{column}{.5\textwidth}
      \textbf{Right Table: DEPARTMENTS}
      
      \medskip
      \begin{center}
        \tiny{
          \begin{tabulary}{\textwidth}{CCC}
            DEPT\_ID & DEPT\_NAME  & DEPT\_FLOOR \\
            \hline
            31       & Sales       & 1           \\
          \end{tabulary}
        }
      \end{center}

    \end{column}

  \end{columns}

  \bigskip
  \begin{center}
    \textbf{Result: } For matched rows, the result set is
    identical to Inner Join.
  \end{center}

\end{frame}

\begin{frame} %%%%%%%%%%%%%%%%%%%%%%%%%%%%%%%%%%%%%%%%%%%%%%%%%%%%%%%%%%%%%%%%%%
  \frametitle{Left Outer Join: Example Result Set (2)}

  \begin{columns}[T]
    \begin{column}{.5\textwidth}
      \textbf{Left Table: EMPLOYEES}
      
      \medskip
      \begin{center}
        \tiny{
          \begin{tabulary}{\textwidth}{RCLLC}
            EID & DEPT\_ID & LAST\_NAME & FIRST\_NAME &            \\
            \hline
            11  & NULL     & Williams   & Serena      & \rightarrow \\
          \end{tabulary}
        }

      \end{center}
    \end{column}
    
    \begin{column}{.5\textwidth}
      \textbf{Right Table: DEPARTMENTS}
      
      \medskip
      \begin{center}
        \tiny{
          \begin{tabulary}{\textwidth}{CLC}
            DEPT\_ID & DEPT\_NAME  & DEPT\_FLOOR \\
            \hline
            NULL     & NULL        & NULL        \\
          \end{tabulary}
        }
      \end{center}

    \end{column}

  \end{columns}

  \bigskip
  \begin{center}
    \textbf{Result: } An Inner Join would have dropped this row.
  \end{center}

\end{frame}

\begin{frame} %%%%%%%%%%%%%%%%%%%%%%%%%%%%%%%%%%%%%%%%%%%%%%%%%%%%%%%%%%%%%%%%%
  \frametitle{Left Outer Join: Result Set}
  \begin{center}
    {\tiny
      \begin{tabulary}{\textwidth}{RCLLCCLC}
        EID & DEPT\_ID & LAST\_NAME & FIRST\_NAME & GENDER & DEPT\_ID & DEPT\_NAME  & DEPT\_FLOOR \\
        \hline
        1   & 31       & Rafferty   & Gerry       & M      & 31       & Sales       & 1           \\
        3   & 33       & Jones      & Jon         & M      & 33       & Engineering & 3           \\
        5   & 33       & Heisenberg & Werner      & M      & 33       & Engineering & 3           \\
        7   & 34       & Robinson   & Elizabeth   & F      & 34       & Clerical    & 2           \\
        9   & 34       & Smith      & Jefferson   & M      & 34       & Clerical    & 2           \\
        11   & NULL     & Williams   & Serena      & NULL   & NULL     & NULL        & NULL        \\
      \end{tabulary}
    }
  \end{center}
\end{frame}

\begin{frame}[fragile]
  \frametitle{Left Outer Join: DANGER!}
    \textbf{Question:} How many rows will the following queries
    return?

    \textbf{Hint:} They return the same number of rows.
  \bigskip

  \begin{lstlisting}[title={\tiny Source: https://github.com/Choens/sql-survival-guide/blob/master/sql/04-joins/left-join.sql}]
    select *
    from EMPLOYEES empl left join DEPARTMENTS dept
    on empl.dept_id = dept.dept_id
    where dept.dept_id < 34
    ;
  \end{lstlisting}

  \begin{lstlisting}[title={\tiny Source: https://github.com/Choens/sql-survival-guide/blob/master/sql/04-joins/left-join.sql}]
    select *
    from EMPLOYEES empl, DEPARTMENTS dept
    where
        empl.dept_id = dept.dept_id 
        and dept.dept_id < 34
    ;
  \end{lstlisting}
  
\end{frame}

\begin{frame}
  \frametitle{Left Outer Join: DANGER!}
  {\huge \textbf{Answer: }Only 3 rows.}

\end{frame}

\begin{frame}[fragile]
  \frametitle{Left Outer Join: DANGER!}
    \textbf{Question:} Why doesn't this return 6 rows?
  \bigskip

  \begin{lstlisting}[title={\tiny Source: https://github.com/Choens/sql-survival-guide/blob/master/sql/04-joins/left-join.sql}]
    select *
    from EMPLOYEES empl left join DEPARTMENTS dept
    on empl.dept_id = dept.dept_id
    where dept.dept_id < 34
    ;
  \end{lstlisting}
  
  \bigskip
  \pause 
  \textbf{Answer:} SQL Order of Operations!
  \begin{itemize}
  \item The FROM clause returns 6 rows.
  \item The WHERE clause runs AFTER the FROM clause, and drops 3 rows.
  \item End Result: 3 rows returned.
  \end{itemize}

\end{frame}

\begin{frame}[fragile]
  \frametitle{Left Outer Join: Danger!}

  \begin{lstlisting}[title={\tiny Source:
      https://github.com/Choens/sql-survival-guide/blob/master/sql/04-joins/left-join.sql}]
    select *
    from (
           -- This sub-query returns 6 rows.
           select *
           from EMPLOYEES empl left join DEPARTMENTS dept
           on empl.dept_id = dept.dept_id
         ) src0
    -- But then ommits three of those rows from the final result set.     
    where src0.dept_id < 34
    ;
  \end{lstlisting}

  \bigskip
  This highlights what the former query is doing. \\
  It is easier to understand why this only returns 3 rows.

\end{frame}

\begin{frame}[fragile]
  \frametitle{Left Outer Join: DANGER!}
    \textbf{Question:} How many rows will the following query return?
  \bigskip

  \begin{lstlisting}[title={\tiny Source: https://github.com/Choens/sql-survival-guide/blob/master/sql/04-joins/left-join.sql}]
    select *
    from EMPLOYEES empl left join DEPARTMENTS dept
    on empl.dept_id = dept.dept_id
       and dept.dept_id < 34
    ;
  \end{lstlisting}
  
  \bigskip
  \pause 
  \textbf{Answer:} 6 rows
  \begin{itemize}
  \item Moved the filter statement to the FROM clause.
  \item It is OK to have a WHERE clause that refers to the columns in
    the Left Table (EMPLOYEES).
  \end{itemize}

\end{frame}



\subsection{Right Outer Join} %%%%%%%%%%%%%%%%%%%%%%%%%%%%%%%%%%%%%%%%%%%%%%%%%

\begin{frame}
  \frametitle{Right Outer Join: Discussion}
  
  \textbf{Right Outer Join:}

  \bigskip
  \begin{itemize}
  \item Basically a left join with the table order reversed.
  \item When a row in the right table does not match any row in the
    left table, there will be NULLS in all columns from the left
    table.
  \item Right Join is the same as Right Outer Join
  \item Speakers of LTR languages tend to prefer Left Joins.
  \item Speakers of RTL languages often prefer  Right Joins.
  \end{itemize}

\end{frame}

\begin{frame}[fragile] %%%%%%%%%%%%%%%%%%%%%%%%%%%%%%%%%%%%%%%%%%%%%%%%%%%%%%%%%
  \frametitle{Right Outer Join: Example}

  \textbf{Question:} How many rows will the following query return?
  \bigskip

  \begin{lstlisting}[title={\tiny Source: https://github.com/Choens/sql-survival-guide/blob/master/sql/04-joins/right-join.sql}]
    select *
    from EMPLOYEES empl right join DEPARTMENTS dept
    on empl.dept_id = dept.dept_id
    ;
  \end{lstlisting}

  \smallskip
  \textbf{Note: }Not supported by some SQL implementations such as SQLite.
  
  \bigskip
  \pause 
  \textbf{Answer:} 6 rows
  
\end{frame}

\begin{frame} %%%%%%%%%%%%%%%%%%%%%%%%%%%%%%%%%%%%%%%%%%%%%%%%%%%%%%%%%%%%%%%%%%
  \frametitle{Right Outer Join: Example Result Set (1)}
  \begin{columns}[T]
    \begin{column}{.5\textwidth}
      \textbf{Left Table: EMPLOYEES}
      
      \medskip
      \begin{center}
        \tiny{
          \begin{tabulary}{\textwidth}{CCCCC}
            EID & DEPT\_ID & LAST\_NAME & FIRST\_NAME \\
            \hline
            1   & 31       & Rafferty   & Gerry       \\
          \end{tabulary}
        }

      \end{center}
    \end{column}
    
    \begin{column}{.5\textwidth}
      \textbf{Right Table: DEPARTMENTS}
      
      \medskip
      \begin{center}
        \tiny{
          \begin{tabulary}{\textwidth}{CCCC}
                       & DEPT\_ID & DEPT\_NAME  & DEPT\_FLOOR \\
            \hline
            \leftarrow & 31       & Sales       & 1           \\
          \end{tabulary}
        }
      \end{center}

    \end{column}

  \end{columns}

  \bigskip
  \begin{center}
    \textbf{Result: } For matched rows, the result set is the same as
    Inner Join.
  \end{center}

\end{frame}

\begin{frame}[fragile] %%%%%%%%%%%%%%%%%%%%%%%%%%%%%%%%%%%%%%%%%%%%%%%%%%%%%%%%%
  \frametitle{Right Outer Join: Example Result Set (2)}
  \begin{columns}[T]
    \begin{column}{.5\textwidth}
      \textbf{Left Table: EMPLOYEES}
      
      \medskip
      \begin{center}
        \tiny{
          \begin{tabulary}{\textwidth}{CCCC}
            EID  & DEPT\_ID & LAST\_NAME & FIRST\_NAME \\
            \hline
            NULL & NULL     & NULL       & NULL         \\
          \end{tabulary}
        }

      \end{center}
    \end{column}
    
    \begin{column}{.5\textwidth}
      \textbf{Right Table: DEPARTMENTS}
      
      \medskip
      \begin{center}
        \tiny{
          \begin{tabulary}{\textwidth}{CCCC}
                       & DEPT\_ID & DEPT\_NAME  & DEPT\_FLOOR \\
            \hline
            \leftarrow & 35       & Marketing   & 3            \\
          \end{tabulary}
        }
      \end{center}

    \end{column}

  \end{columns}

  \bigskip
  \textbf{Result: } 
  \begin{itemize}
  \item The Right Join keeps Dept 35 in the result set.
  \item The same thing can be written as a Left Join:
  \end{itemize} 

  \begin{lstlisting}[title={\tiny Source: https://github.com/Choens/sql-survival-guide/blob/master/sql/04-joins/right-join.sql}]
    select *
    from DEPARTMENTS dept left join EMPLOYEES empl
    on dept.dept_id = empl.dept_id
    ;
  \end{lstlisting}

\end{frame}

\begin{frame}[fragile] %%%%%%%%%%%%%%%%%%%%%%%%%%%%%%%%%%%%%%%%%%%%%%%%%%%%%%%%
  \frametitle{Right Outer Join: Result Set}
  \begin{center}
    {\tiny
      \begin{tabulary}{\textwidth}{RCLLCCLC}
        EID  & DEPT\_ID & LAST\_NAME & FIRST\_NAME & GENDER & DEPT\_ID & DEPT\_NAME  & DEPT\_FLOOR \\
        \hline
        1    & 31       & Rafferty   & Gerry       & M      & 31       & Sales       & 1           \\
        3    & 33       & Jones      & Jon         & M      & 33       & Engineering & 3           \\
        5    & 33       & Heisenberg & Werner      & M      & 33       & Engineering & 3           \\
        7    & 34       & Robinson   & Elizabeth   & F      & 34       & Clerical    & 2           \\
        9    & 34       & Smith      & Jefferson   & M      & 34       & Clerical    & 2           \\
        NULL & NULL     & NULL       & NULL        & NULL   & 35       & Marketing   & 3           \\
      \end{tabulary}
    }
  \end{center}

  \bigskip
  \textbf{Question: }What happened to Serena Williams?

  \bigskip
  \pause  
  \textbf{Answer: } Because DEPT\_ID is NULL, she is not in the result
  set.

\end{frame}

\begin{frame}
  \frametitle{Right Outer Join: Danger!}
  
  \textbf{Danger Will Robinson! Danger!}
  \begin{itemize}
  \item Right Outer Joins are subject to the same dangers as Left
    Outer Joins.
  \item Conceptualizing Right Joins can be hard for native English speakers.
  \item Using left and right joins in the same query is TROUBLE\@.
  \end{itemize}
\end{frame}

\subsection{Full Outer Join} %%%%%%%%%%%%%%%%%%%%%%%%%%%%%%%%%%%%%%%%%%%%%%%%%%

\begin{frame}
  \frametitle{Full Outer Join: Discussion}
  
  \textbf{Full Outer Join:}

  \begin{itemize}
  \item Combines the effect of applying both left and right outer
    joins.
  \item Result set includes all rows from both tables, at least once.
  \item Where rows do not match, the result set will have NULL values
    for every column of the table that lacks a matching row
  \item Not frequently used.
  \item Can be used to understand why a result set is smaller than
    expected.
  \end{itemize}

\end{frame}

\begin{frame}[fragile] %%%%%%%%%%%%%%%%%%%%%%%%%%%%%%%%%%%%%%%%%%%%%%%%%%%%%%%%%
  \frametitle{Full Outer Join: Example}

  \textbf{Question:} How many rows will the following query return?

  \begin{lstlisting}[title={\tiny Source: https://github.com/Choens/sql-survival-guide/blob/master/sql/04-joins/full-outer-join.sql}]
    select *
    from EMPLOYEES empl full outer join DEPARTMENTS dept
    on empl.dept_id = dept.dept_id
    ;
  \end{lstlisting}

  \bigskip
  \textbf{Note: }Not supported by some SQL implementations such as
  SQLite.

  \bigskip
  \pause
  \textbf{Answer: }7 rows.
 
\end{frame}

\begin{frame} %%%%%%%%%%%%%%%%%%%%%%%%%%%%%%%%%%%%%%%%%%%%%%%%%%%%%%%%%%%%%%%%%%
  \frametitle{Full Outer Join: Example Result Set (1)}
  
  \begin{columns}[T]
    \begin{column}{.5\textwidth}
      \textbf{Left Table: EMPLOYEES}
      
      \medskip
      \begin{center}
        \tiny{
          \begin{tabulary}{\textwidth}{CCCCC}
            EID & DEPT\_ID & LAST\_NAME & FIRST\_NAME &            \\
            \hline
            1   & 31       & Rafferty   & Gerry       & \rightarrow \\
          \end{tabulary}
        }

      \end{center}
    \end{column}
    
    \begin{column}{.5\textwidth}
      \textbf{Right Table: DEPARTMENTS}
      
      \medskip
      \begin{center}
        \tiny{
          \begin{tabulary}{\textwidth}{CCCC}
                       & DEPT\_ID & DEPT\_NAME  & DEPT\_FLOOR \\
            \hline
            \leftarrow & 31       & Sales       & 1           \\
          \end{tabulary}
        }
      \end{center}

    \end{column}

  \end{columns}

  \bigskip
  \textbf{Result:}
  \begin{itemize}
  \item For matched rows, result set is same as Inner Join.
  \end{itemize}

\end{frame}

\begin{frame} %%%%%%%%%%%%%%%%%%%%%%%%%%%%%%%%%%%%%%%%%%%%%%%%%%%%%%%%%%%%%%%%%%
  \frametitle{Full Outer Join: Example Result Set (2)}
  
  \begin{columns}[T]
    \begin{column}{.5\textwidth}
      \textbf{Left Table: EMPLOYEES}
      
      \medskip
      \begin{center}
        \tiny{
          \begin{tabulary}{\textwidth}{CCCCC}
            EID & DEPT\_ID & LAST\_NAME & FIRST\_NAME &            \\
            \hline
            11  & NULL     & Williams   & Serena      & \rightarrow \\
          \end{tabulary}
        }

      \end{center}
    \end{column}
    
    \begin{column}{.5\textwidth}
      \textbf{Right Table: DEPARTMENTS}
      
      \medskip
      \begin{center}
        \tiny{
          \begin{tabulary}{\textwidth}{CCCC}
                       & DEPT\_ID & DEPT\_NAME  & DEPT\_FLOOR \\
            \hline
            \leftarrow & NULL     & NULL        & NULL         \\
          \end{tabulary}
        }
      \end{center}

    \end{column}

  \end{columns}

  \bigskip
  \textbf{Result:}
  \begin{itemize}
  \item For unmatched rows: KEEP THEM ALL!
  \end{itemize}
\end{frame}

\begin{frame} %%%%%%%%%%%%%%%%%%%%%%%%%%%%%%%%%%%%%%%%%%%%%%%%%%%%%%%%%%%%%%%%%%
  \frametitle{Full Outer Join: Example Result Set (3)}
  
  \begin{columns}[T]
    \begin{column}{.5\textwidth}
      \textbf{Left Table: EMPLOYEES}
      
      \medskip
      \begin{center}
        \tiny{
          \begin{tabulary}{\textwidth}{CCCCC}
            EID  & DEPT\_ID & LAST\_NAME & FIRST\_NAME &             \\
            \hline
            NULL & NULL     & NULL       & NULL        & \rightarrow \\
          \end{tabulary}
        }

      \end{center}
    \end{column}
    
    \begin{column}{.5\textwidth}
      \textbf{Right Table: DEPARTMENTS}
      
      \medskip
      \begin{center}
        \tiny{
          \begin{tabulary}{\textwidth}{CCCC}
                       & DEPT\_ID & DEPT\_NAME  & DEPT\_FLOOR \\
            \hline
            \leftarrow & 35       & Marketing       & 3       \\
          \end{tabulary}
        }
      \end{center}

    \end{column}

  \end{columns}

  \bigskip
  \textbf{Result:}
  \begin{itemize}
  \item For unmatched rows: KEEP THEM ALL!
  \end{itemize}

\end{frame}

\begin{frame} %%%%%%%%%%%%%%%%%%%%%%%%%%%%%%%%%%%%%%%%%%%%%%%%%%%%%%%%%%%%%%%%%
  \frametitle{Full Outer Join: Result Set}
  \begin{center}
    {\tiny
      \begin{tabulary}{\textwidth}{RCLLCCLC}
        EID & DEPT\_ID & LAST\_NAME & FIRST\_NAME & GENDER & DEPT\_ID & DEPT\_NAME  & DEPT\_FLOOR \\
        \hline
        1    & 31       & Rafferty   & Gerry       & M      & 31       & Sales       & 1           \\
        3    & 33       & Jones      & Jon         & M      & 33       & Engineering & 3           \\
        5    & 33       & Heisenberg & Werner      & M      & 33       & Engineering & 3           \\
        7    & 34       & Robinson   & Elizabeth   & F      & 34       & Clerical    & 2           \\
        9    & 34       & Smith      & Jefferson   & M      & 34       & Clerical    & 2           \\
        11   & NULL     & Williams   & Serena      & NULL   & NULL     & NULL        & NULL        \\
        NULL & NULL     & NULL       & NULL        & NULL   & 35       & Marketing   & 3           \\
      \end{tabulary}
    }
  \end{center}
\end{frame}

\section{Self Join} %%%%%%%%%%%%%%%%%%%%%%%%%%%%%%%%%%%%%%%%%%%%%%%%%%%%%%%%%%%

\begin{frame}
  \frametitle{Self Join: Discussion}
  \textbf{Self Join:}

  \begin{itemize}
  \item Joins a table to itself.
  \item To do so, use one of the Joins we have already discussed.
  \item This is more useful than you might think.
  \item Often used in conjunction with a subquery.
  \item To demo, we will need a new table.
  \end{itemize}
\end{frame}

\begin{frame} %%%%%%%%%%%%%%%%%%%%%%%%%%%%%%%%%%%%%%%%%%%%%%%%%%%%%%%%%%%%%%%%%%
  \frametitle{Table: SALES}

  \begin{center}
    \begin{tabulary}{\textwidth}{CLCC}
      SALE\_ID & EID  & SALE\_VAL & SALE\_DT    \\
      \hline
      1        & 1    & 15.00     & 2015-01-01 \\
      2        & 1    & 30.12     & 2015-06-15 \\
      3        & 1    & 45.79     & 2015-03-02 \\
    \end{tabulary}

    \bigskip
    \textbf{Aliased as `sale'.}

  \end{center}

  \bigskip
  \textbf{Note:} Query to create this table is not in
  create-tables.sql (yet).

\end{frame}

\begin{frame}[fragile]
  \frametitle{Self Join: Example}

  \textbf{Question:} How many rows will the following query return?

  \begin{lstlisting}[title={\tiny Source: https://github.com/Choens/sql-survival-guide/blob/master/sql/04-joins/self-join.sql}]
    select sale.eid, sale.eid, sale.sale_val, sale_dt
    from SALES sale 
    inner join (
                 -- Last Sale (ls)
                 -- This query runs BEFORE the Join Clause.
                 select eid, max(sale_dt) max_dt 
                 from SALES 
                 group by eid
               ) ls
    on sale.eid = ls.eid and sale.sale_dt = ls.max_dt
    ;
  \end{lstlisting}

  \bigskip
  \pause
  \textbf{Answer:} Only 1 row.

\end{frame}

\begin{frame} %%%%%%%%%%%%%%%%%%%%%%%%%%%%%%%%%%%%%%%%%%%%%%%%%%%%%%%%%%%%%%%%%%
  \frametitle{Self Join: Example Result Set}

  \bigskip
  \begin{columns}[T]
    \begin{column}{.5\textwidth}
      \textbf{Left Table: SALES}
      
      \medskip
      \begin{center}
        \tiny{
          \begin{tabulary}{\textwidth}{CCCCC}
            SALE\_ID & EID  & SALE\_VAL & SALE\_DT &    \\
            \hline
            1        & 1    & 15.00     & 2015-01-01 & \rightarrow \\
          \end{tabulary}
        }

      \end{center}
    \end{column}
    
    \begin{column}{.5\textwidth}
      \textbf{Right Table: LAST SALE}
      
      \medskip
      \begin{center}
        \tiny{
          \begin{tabulary}{\textwidth}{CCC}
                       & EID & MAX\_DT    \\
            \hline
            \leftarrow & 1   & 2015-06-15 \\
          \end{tabulary}
        }
      \end{center}

    \end{column}

  \end{columns}

  \bigskip
  \textbf{Result:}
  \begin{itemize}
  \item The sub-query (LAST SALE) is run first. It returns 1 row.
  \item Inner Join returns matching rows between SALE and LAST
    SALE.
  \item Each copy of a table must have a unique alias.
  \item A subquery is not required, but is common.
  \end{itemize}

\end{frame}

\begin{frame} %%%%%%%%%%%%%%%%%%%%%%%%%%%%%%%%%%%%%%%%%%%%%%%%%%%%%%%%%%%%%%%%%
  \frametitle{Self Outer Join: Result Set}
  \begin{center}
    {\small
      \begin{tabulary}{\textwidth}{CLCC}
        SALE\_ID & EID  & SALE\_VAL & SALE\_DT    \\
        \hline
        2        & 1    & 30.12     & 2015-06-15 \\
      \end{tabulary}
    }
  \end{center}
  
\end{frame}

\section{Aggregation} %%%%%%%%%%%%%%%%%%%%%%%%%%%%%%%%%%%%%%%%%%%%%%%%%%%%%%%%%%

\begin{frame}
  \frametitle{Aggregation: How To}

  Separate your thoughts into two steps.:

  \begin{enumerate}
  \item FROM
  \item GROUP BY
  \end{enumerate}
  
  \bigskip
  \begin{center}
  \textbf{Pro Tip: }Treat each step in the SQL Order of Operations as though it has an
  independent result set.
  \end{center}
\end{frame}

\section{Get Relational} %%%%%%%%%%%%%%%%%%%%%%%%%%%%%%%%%%%%%%%%%%%%%%%%%%%%%%%

\begin{frame}
  \frametitle{Relational Data: How To}
  
  Working with relational data is harder than working with flat data.

  \bigskip
  Some things to consider:

  \begin{itemize}
  \item What is the `unit' of your table(s)?
  \item Valid data must have a unique identifier for each row in a table.
    \begin{itemize}
    \item People / Employees / Medicaid Recipients
    \item Departments / Providers
    \item Events / MDC Codes / Medications
    \end{itemize}
  \item The unique identifier changes across tables.
    \begin{itemize}
    \item One to One
    \item One to Many
    \item To use, you must understand how the tables are related.
    \end{itemize}
  \end{itemize}

\end{frame}

\begin{frame}
  \frametitle{Relational Data: Practical How To}
  
  Assume you are joining two tables (Table A, Table B):

  \smallskip
  Table A:
  \begin{itemize}
  \item Count the number of total records in Table A.
  \item Count the number of distinct unique identifiers in Table A.
  \item If they don't match. That isn't the unique identifier.
  \item Look for Nulls in any column you are going to filter, sort or
    join by.
  \end{itemize}

  \smallskip
  Table B:
  \begin{itemize}  
  \item Count the number of total records in Table B.
  \item Count the number of distinct unique identifiers in Table B.
  \item If they don't match. That isn't the unique identifier.
  \item Look for Nulls in any column you are going to filter, sort or
    join by.
  \end{itemize}

\end{frame}

\begin{frame}
  \frametitle{Approach Relational Data Like a Scientist (1)}
  
  \textbf{Don't Assume, Prove:}
  \begin{itemize}
  \item You believe Table A has a one-to-many-relationship with Table
    B.
  \item Don't assume you know what is going on.
  \item We are scientists. 
  \item We have tools for this.
  \end{itemize}

\end{frame}

\begin{frame}
  \frametitle{Approach Relational Data Like a Scientist (2)}
    
  \textbf{Science Tools For SQL:}
  \begin{itemize}
  \item Hypothesis: Every row in Table A matches one (+) rows in
    Table B.
  \item Test your hypothesis. Is it true?
  \item Are you dropping any rows from A?
  \item Are you duplicating unexpectedly?
  \end{itemize}

  \bigskip
  \textbf{Pro Tip: }Build a complex query piece by piece! (Test each piece.)
\end{frame}
\section{ Wrap It Up} %%%%%%%%%%%%%%%%%%%%%%%%%%%%%%%%%%%%%%%%%%

\begin{frame}
  \frametitle{Additional Information}
  \begin{itemize}
  \item https://en.wikipedia.org/wiki/Join\_(SQL)
  \end{itemize}
\end{frame}

\begin{frame}
  \frametitle{Questions?}

  \begin{center}
    {\large That was a lot of information. And there is a lot of text
      silly text on this slide. If you are still reading this,
      congrats. You must be wide awake by now. And maybe, you are
      wondering why I wrote all of this.}

    \bigskip
    {\huge  Who has questions?}
  \end{center}

\end{frame}

\end{document}
