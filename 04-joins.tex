\documentclass{beamer}
% 
% Choose how your presentation looks.
% 
% For more themes, color themes and font themes, see:
% http://deic.uab.es/~iblanes/beamer_gallery/index_by_theme.html
% 
\mode<presentation>{
%  \usetheme{Madrid}      % or try Darmstadt, Madrid, Warsaw, ...
  \usetheme{Boadilla}
  \usecolortheme{beaver} % or try albatross, beaver, crane, ...
  \usefonttheme{default}  % or try serif, structurebold, ...
  \setbeamertemplate{navigation symbols}{}
  \setbeamertemplate{caption}[numbered]
} 

\usepackage{graphicx}
\usepackage[english]{babel}
\usepackage[utf8x]{inputenc}
\usepackage{tabulary}

%% For Pretty In-Line Code:
\usepackage{listings}
\usepackage{color}

\definecolor{dkgreen}{rgb}{0,0.6,0}
\definecolor{gray}{rgb}{0.95,0.95,0.95}
\definecolor{mauve}{rgb}{0.58,0,0.82}

\lstdefinestyle{customsql}{
  xleftmargin=4ex
  ,language=SQL
}


\lstset{
  frame=none,
  language=SQL,
  aboveskip=3mm,
  belowskip=3mm,
  showstringspaces=false,
  columns=flexible,
  basicstyle={\small\ttfamily},
  numbers=left,
  numberstyle=\tiny\color{black},
  keywordstyle=\color{blue},
  commentstyle=\color{dkgreen},
  backgroundcolor=\color{gray},
  stringstyle=\color{mauve},
  breaklines=true,
  breakatwhitespace=true,
  tabsize=2,
  captionpos=b,
  escapechar=@,
  style=customsql
}

\AtBeginSection[]{
  \begin{frame}<beamer>
    \frametitle{Outline for section \thesection}
    \tableofcontents[currentsection]
  \end{frame}
}

\title[SQL Survival Guide]{SQL Survival Guide\\Joins}
\author{Andy Choens}
\institute{EBCoP}
\date{May 18, 2015}

\begin{document}

\begin{frame} %%%%%%%%%%%%%%%%%%%%%%%%%%%%%%%%%%%%%%%%%%%%%%%%%%%%%%%%%%%%%%%%%%
  \titlepage
\end{frame}

\begin{frame}{Outline} %%%%%%%%%%%%%%%%%%%%%%%%%%%%%%%%%%%%%%%%%%%%%%%%%%%%%%%%%
  \tableofcontents
\end{frame}



\section{Example Data} %%%%%%%%%%%%%%%%%%%%%%%%%%%%%%%%%%%%%%%%%%%%%%%%%%%%%%%%%

\begin{frame} %%%%%%%%%%%%%%%%%%%%%%%%%%%%%%%%%%%%%%%%%%%%%%%%%%%%%%%%%%%%%%%%%%
  \frametitle{Example Data}

  \begin{itemize}
  \item Two tables of example data!
  \item ALL Code provided: sql/04-joins
    \begin{itemize}
    \item create\_tables.sql: Creates the example tables.
    \item Everything Else: Other files contain example queries.
    \end{itemize}
  \item Example data in SQLite: data/04-joins.sqlite.
  \end{itemize}

\end{frame}

\begin{frame} %%%%%%%%%%%%%%%%%%%%%%%%%%%%%%%%%%%%%%%%%%%%%%%%%%%%%%%%%%%%%%%%%%
  \frametitle{Table: Departments}

  \begin{center}
    \begin{tabulary}{\textwidth}{CLC}
      DEPT\_ID & DEPT\_NAME  & DEPT\_FLOOR \\
      \hline
      31       & Sales       & 1           \\
      33       & Engineering & 3           \\
      34       & Clerical    & 2           \\
      35       & Marketing   & 3           \\
    \end{tabulary}

      \bigskip
      \textbf{Aliased as `dept'.}

  \end{center}

\end{frame}

\begin{frame} %%%%%%%%%%%%%%%%%%%%%%%%%%%%%%%%%%%%%%%%%%%%%%%%%%%%%%%%%%%%%%%%%%
  \frametitle{Table: Employees}

  \begin{center}
    \begin{tabulary}{\textwidth}{RCLLC}
      EID & DEPT\_ID & LAST\_NAME & FIRST\_NAME & GENDER\\
      \hline
      1   & 31       & Rafferty   & Gerry       & M     \\
      3   & 33       & Jones      & Jon         & M     \\
      5   & 33       & Heisenberg & Werner      & M     \\
      7   & 34       & Robinson   & Elizabeth   & F     \\
      9   & 34       & Smith      & Jefferson   & M     \\        
      11  & NULL     & Williams   & Serena      & F     \\
   \end{tabulary}

     \bigskip
     \textbf{Aliased as `empl'.}

  \end{center}
  
\end{frame}



\section{Dafynitions} %%%%%%%%%%%%%%%%%%%%%%%%%%%%%%%%%%%%%%%%%%%%%%%%%%%%%%%%%%

\begin{frame} %%%%%%%%%%%%%%%%%%%%%%%%%%%%%%%%%%%%%%%%%%%%%%%%%%%%%%%%%%%%%%%%%%
  \frametitle{Dafynitions?}

  We need a common set of terms to discuss this.
  
  TODO: The slides in this section need a lot of work.

\end{frame}  

\begin{frame} %%%%%%%%%%%%%%%%%%%%%%%%%%%%%%%%%%%%%%%%%%%%%%%%%%%%%%%%%%%%%%%%%%
  \frametitle{Dafynition: Relational Data}

  TODO

\end{frame}  

\begin{frame} %%%%%%%%%%%%%%%%%%%%%%%%%%%%%%%%%%%%%%%%%%%%%%%%%%%%%%%%%%%%%%%%%%
  \frametitle{Dafynition: Normalized Data}

  TODO

\end{frame}  

\begin{frame} %%%%%%%%%%%%%%%%%%%%%%%%%%%%%%%%%%%%%%%%%%%%%%%%%%%%%%%%%%%%%%%%%%
  \frametitle{Dafynition: Result Set}

  TODO

\end{frame}  

\begin{frame} %%%%%%%%%%%%%%%%%%%%%%%%%%%%%%%%%%%%%%%%%%%%%%%%%%%%%%%%%%%%%%%%%%
  \frametitle{Dafynition: Clause}
  
  The following protected SQL commands START a clause:
  \smallskip
  \begin{itemize}
  \item SELECT
  \item FROM
  \item WHERE
  \item HAVING
  \end{itemize}

  \bigskip
  \pause
  Style:
  \smallskip
  \begin{itemize}
  \item Align SQL Clauses
  \item Predicates should be indented consistently.
  \item Feel free to ask me why I structure SQL queries the way I do.
  \end{itemize}
  
\end{frame}  

\begin{frame} %%%%%%%%%%%%%%%%%%%%%%%%%%%%%%%%%%%%%%%%%%%%%%%%%%%%%%%%%%%%%%%%%%
  \frametitle{Dafynition: Join Clause?}

  \begin{itemize}
  \item The FIRST thing you should write.
  \item Combines records from two or more tables (result set).
  \item SQL Joins are a difficult skill to master.
  \item They are necessary for working with normalized, relational
    data.
  \end{itemize}

\end{frame}  



\section{SQL Order Of Operations} %%%%%%%%%%%%%%%%%%%%%%%%%%%%%%%%%%%%%%%%%%%%%%

\begin{frame} %%%%%%%%%%%%%%%%%%%%%%%%%%%%%%%%%%%%%%%%%%%%%%%%%%%%%%%%%%%%%%%%%%
  \frametitle{Order of Operations}

  What is the answer to this (silly) equation?

  \bigskip
  \[\sqrt{(2^2+2) \cdot 6}\]

  \bigskip
  \pause
  And how were we all able to come up with the same answer?

\end{frame}

\begin{frame} %%%%%%%%%%%%%%%%%%%%%%%%%%%%%%%%%%%%%%%%%%%%%%%%%%%%%%%%%%%%%%%%%%
  \frametitle{Order of Operations}

  Just like math, SQL has an order of operations: 

  \medskip
  \begin{enumerate}
  \item FROM
  \item WHERE
  \item GROUP BY
  \item SELECT
  \item HAVING
  \item ORDER BY
  \end{enumerate}

\end{frame}

\begin{frame} %%%%%%%%%%%%%%%%%%%%%%%%%%%%%%%%%%%%%%%%%%%%%%%%%%%%%%%%%%%%%%%%%%
  \frametitle{Order of Operations}

  Some last order of operation notes:

  \bigskip
  \begin{itemize}
  \item Sub-queries are run before the outer query.
  \item The optimizer may reorganize query (relational algebra).
  \item Sometimes writing things out of order can bite you.
  \end{itemize}
  
\end{frame}

\section{ANSI  Joins} %%%%%%%%%%%%%%%%%%%%%%%%%%%%%%%%%%%%%%%%%%%%%%%%%%%%%%%%%%

\begin{frame} %%%%%%%%%%%%%%%%%%%%%%%%%%%%%%%%%%%%%%%%%%%%%%%%%%%%%%%%%%%%%%%%%%
  \frametitle{Types of Joins}

  There are \textbf{only} 5 ANSI joins:

  \smallskip

  \begin{columns}

    \begin{column}{.5\textwidth}
      \begin{enumerate}
      \item Cross Join
      \item Inner Join
      \item Left Outer Join
      \item Right Outer Join
      \item Full Outer Join
      \end{enumerate}
    \end{column}
    
    \begin{column}{.5\textwidth}
        \begin{enumerate}
        \item DO NOT USE!
        \item Very Useful.
        \item Very Useful.
        \item Most of you should not use this.
        \item You will rarely use this.
        \end{enumerate}
    \end{column}

  \end{columns}

  \bigskip
  \textbf{Most SQL queries can be written with only 2 types of joins.}

\end{frame}  

\subsection{Cross Joins} %%%%%%%%%%%%%%%%%%%%%%%%%%%%%%%%%%%%%%%%%%%%%%%%%%%%%%%

\begin{frame} %%%%%%%%%%%%%%%%%%%%%%%%%%%%%%%%%%%%%%%%%%%%%%%%%%%%%%%%%%%%%%%%%%
  \frametitle{Cross Join: Discussion}
  \begin{itemize}
  \item Returns the Cartesian product of the tables in the FROM
    clause. 
  \item Can be written explicitly and implicitly.
  \item This is \textbf{ALMOST ALWAYS} a bad idea.
  \end{itemize}
\end{frame}

\begin{frame}[fragile] %%%%%%%%%%%%%%%%%%%%%%%%%%%%%%%%%%%%%%%%%%%%%%%%%%%%%%%%%
  \frametitle{Explicit Cross Join: Example}

  \textbf{Question:}\\ How many rows will the following query return?
  \bigskip

  \begin{lstlisting}[title={\tiny Source: https://github.com/Choens/sql-survival-guide/blob/master/sql/04-joins/cross-joins.sql}]
   select *
   from EMPLOYEES empl cross join DEPARTMENTS dept
   ;
  \end{lstlisting}
  
  \bigskip
  \pause 
  \textbf{Answer:}\\ 24 rows
  
\end{frame}

\begin{frame}
  \frametitle{Cross Join: What Happens}
  TODO: A graphical representation of what it is doing.
\end{frame}

\begin{frame} %%%%%%%%%%%%%%%%%%%%%%%%%%%%%%%%%%%%%%%%%%%%%%%%%%%%%%%%%%%%%%%%%%
  \frametitle{Cross Join: Result Set}

  \begin{center}
    {\tiny
      \begin{tabulary}{\textwidth}{RCLLCRLC}
        EID & DEPT\_ID & LAST\_NAME & FIRST\_NAME & GENDER & DEPT\_ID & DEPT\_NAME  & DEPT\_FLOOR \\
        \hline
        1   & 31       & Rafferty   & Gerry       & M      & 31       & Sales       & 1           \\
        1   & 31       & Rafferty   & Gerry       & M      & 33       & Engineering & 3           \\
        1   & 31       & Rafferty   & Gerry       & M      & 34       & Clerical    & 2           \\
        1   & 31       & Rafferty   & Gerry       & M      & 35       & Marketing   & 3           \\
        3   & 33       & Jones      & Jon         & M      & 31       & Sales       & 1           \\
        3   & 33       & Jones      & Jon         & M      & 33       & Engineering & 3           \\
        3   & 33       & Jones      & Jon         & M      & 34       & Clerical    & 2           \\
        3   & 33       & Jones      & Jon         & M      & 35       & Marketing   & 3           \\
        5   & 33       & Heisenberg & Werner      & M      & 31       & Sales       & 1           \\
        5   & 33       & Heisenberg & Werner      & M      & 33       & Engineering & 3           \\
        5   & 33       & Heisenberg & Werner      & M      & 34       & Clerical    & 2           \\
        5   & 33       & Heisenberg & Werner      & M      & 35       & Marketing   & 3           \\
        7   & 34       & Robinson   & Elizabeth   & F      & 31       & Sales       & 1           \\
        7   & 34       & Robinson   & Elizabeth   & F      & 33       & Engineering & 3           \\
        7   & 34       & Robinson   & Elizabeth   & F      & 34       & Clerical    & 2           \\
        7   & 34       & Robinson   & Elizabeth   & F      & 35       & Marketing   & 3           \\
        9   & 34       & Smith      & Jefferson   & M      & 31       & Sales       & 1           \\
        9   & 34       & Smith      & Jefferson   & M      & 33       & Engineering & 3           \\
        9   & 34       & Smith      & Jefferson   & M      & 34       & Clerical    & 2           \\
        9   & 34       & Smith      & Jefferson   & M      & 35       & Marketing   & 3           \\
        11  & [NULL]   & Williams   & Serena      & F      & 31       & Sales       & 1           \\
        11  & [NULL]   & Williams   & Serena      & F      & 33       & Engineering & 3           \\
        11  & [NULL]   & Williams   & Serena      & F      & 34       & Clerical    & 2           \\
        11  & [NULL]   & Williams   & Serena      & F      & 35       & Marketing   & 3           \\
      \end{tabulary}
    }
  \end{center}

\end{frame}

\begin{frame}[containsverbatim] %%%%%%%%%%%%%%%%%%%%%%%%%%%%%%%%%%%%%%%%%%%%%%%%
  \frametitle{Implicit Cross Join: Example}
 
  \begin{lstlisting}[title={\tiny Source: https://github.com/Choens/sql-survival-guide/blob/master/sql/04-joins/cross-joins.sql}]
    select *
    from EMPLOYEES empl, DEPARTMENTS dept
    ;
  \end{lstlisting}
  \bigskip
  \begin{itemize}
  \item Also returns 24 rows.
  \item What does this query look like?
  \end{itemize}
  
\end{frame}

\begin{frame} %%%%%%%%%%%%%%%%%%%%%%%%%%%%%%%%%%%%%%%%%%%%%%%%%%%%%%%%%%%%%%%%%%
  \frametitle{Cross Join: Results}
  \begin{itemize}
  \item Cross Joins Are Dangerous! (Especially the implicit ones.)
  \item They return the maximum number of rows possible.
  \end{itemize}
\end{frame}



\subsection{Inner Joins} %%%%%%%%%%%%%%%%%%%%%%%%%%%%%%%%%%%%%%%%%%%%%%%%%%%%%%%

\begin{frame}[containsverbatim] %%%%%%%%%%%%%%%%%%%%%%%%%%%%%%%%%%%%%%%%%%%%%%%%
  \frametitle{Inner Join: Discussion}
  \begin{itemize}
  \item Returns all records which have matching records in both
    tables, according to the join-predicate or WHERE clause.
  \item Can be written explicitly or implicitly.
  \item Returns the least number of rows.
  \end{itemize}
\end{frame}

\begin{frame}[fragile] %%%%%%%%%%%%%%%%%%%%%%%%%%%%%%%%%%%%%%%%%%%%%%%%%%%%%%%%%
  \frametitle{Explicit Inner Join: Example}

  \textbf{Question:}\\ How many rows will the following query return?
  \bigskip

  \begin{lstlisting}[title={\tiny Source: https://github.com/Choens/sql-survival-guide/blob/master/sql/04-joins/cross-joins.sql}]
    select *
    from EMPLOYEES empl inner join DEPARTMENTS dept
    on empl.dept_id = dept.dept_id
    ;
  \end{lstlisting}

  \bigskip
  \pause
  \textbf{Answer:}\\ 5 rows

\end{frame}

\begin{frame}
  \frametitle{Inner Join: What Happens}
  TODO: A pictoral representation.
\end{frame}

 \begin{frame} %%%%%%%%%%%%%%%%%%%%%%%%%%%%%%%%%%%%%%%%%%%%%%%%%%%%%%%%%%%%%%%%%
   \frametitle{Inner Join: Result Set}
     \begin{center}
    {\tiny
      \begin{tabulary}{\textwidth}{RCLLCCLC}
        EID & DEPT\_ID & LAST\_NAME & FIRST\_NAME & GENDER & DEPT\_ID & DEPT\_NAME  & DEPT\_FLOOR \\
        \hline
        1   & 31       & Rafferty   & Gerry       & M      & 31       & Sales       & 1           \\
        3   & 33       & Jones      & Jon         & M      & 33       & Engineering & 3           \\
        5   & 33       & Heisenberg & Werner      & M      & 33       & Engineering & 3           \\
        7   & 34       & Robinson   & Elizabeth   & F      & 34       & Clerical    & 2           \\
        9   & 34       & Smith      & Jefferson   & M      & 34       & Clerical    & 2           \\
      \end{tabulary}
    }
  \end{center}
 \end{frame}

\begin{frame}[fragile] %%%%%%%%%%%%%%%%%%%%%%%%%%%%%%%%%%%%%%%%%%%%%%%%%%%%%%%%%
  \frametitle{Implicit Inner Join: Example}

  This should look familiar:
  \bigskip

  \begin{lstlisting}[title={\tiny Source: https://github.com/Choens/sql-survival-guide/blob/master/sql/04-joins/cross-joins.sql}]
    select *
    from EMPLOYEES empl, DEPARTMENTS dept
    where empl.dept_id = dept.dept_id
    ;
  \end{lstlisting}

  \bigskip
  \begin{itemize}
  \item Also returns 5 rows.
  \end{itemize}

\end{frame}

\begin{frame} %%%%%%%%%%%%%%%%%%%%%%%%%%%%%%%%%%%%%%%%%%%%%%%%%%%%%%%%%%%%%%%%%%
  \frametitle{Why Not Use Implicit Join Syntax?}

  \textbf{Question:}\\ Implicit Cross Join v Implicit INNER JOIN: What's the difference?
  
  \bigskip
  \pause
  \textbf{Answer:}\\ The WHERE clause.
  
  \bigskip
  \begin{itemize}
  \item Implicit Join syntax \emph{is} deprecated.
  \item This style makes it too easy to write a Cross Join (cartesian).
  \item It makes it harder to learn the SQL Order of Operations.
  \end{itemize}

  
\end{frame}



\subsection{Left Outer Joins} %%%%%%%%%%%%%%%%%%%%%%%%%%%%%%%%%%%%%%%%%%%%%%%%%%

\begin{frame} %%%%%%%%%%%%%%%%%%%%%%%%%%%%%%%%%%%%%%%%%%%%%%%%%%%%%%%%%%%%%%%%%%
  \frametitle{Outer Joins}
  \begin{itemize}
  \item The result set from an Inner Joins includes matching records ONLY.
  \item The result set from an Outer Join retains more records.
  \item Types of Outer Joins:
    \begin{itemize}
    \item Left Outer Join (Left Join)
    \item Right Outer Join (Right Join)
    \item Full Outer Join
    \end{itemize}
  \item The only difference is which records become part of the result
    set.
  \end{itemize}
\end{frame}

\begin{frame} %%%%%%%%%%%%%%%%%%%%%%%%%%%%%%%%%%%%%%%%%%%%%%%%%%%%%%%%%%%%%%%%%%
  \frametitle{Left Outer Join: Discussion}
  \begin{itemize}
  \item Result set includes all members of the `left' table.
  \end{itemize}
\end{frame}

\begin{frame}[fragile] %%%%%%%%%%%%%%%%%%%%%%%%%%%%%%%%%%%%%%%%%%%%%%%%%%%%%%%%%
  \frametitle{Left Outer Join: Example} 
  
  \textbf{Question:}\\ How many rows will the following query return?
  \bigskip

  \begin{lstlisting}[title={\tiny Source: https://github.com/Choens/sql-survival-guide/blob/master/sql/04-joins/left-joins.sql}]
   select *
   from EMPLOYEES empl left join DEPARTMENTS dept
   on empl.department_id = dept_dept_id
   ;
  \end{lstlisting}
  
  \bigskip
  \pause 
  \textbf{Answer:}\\ 6 rows

\end{frame}

 \begin{frame} %%%%%%%%%%%%%%%%%%%%%%%%%%%%%%%%%%%%%%%%%%%%%%%%%%%%%%%%%%%%%%%%%
   \frametitle{Left Outer Join: Result Set}
     \begin{center}
    {\tiny
      \begin{tabulary}{\textwidth}{RCLLCCLC}
        EID & DEPT\_ID & LAST\_NAME & FIRST\_NAME & GENDER & DEPT\_ID & DEPT\_NAME  & DEPT\_FLOOR \\
        \hline
        1   & 31       & Rafferty   & Gerry       & M      & 31       & Sales       & 1           \\
        3   & 33       & Jones      & Jon         & M      & 33       & Engineering & 3           \\
        5   & 33       & Heisenberg & Werner      & M      & 33       & Engineering & 3           \\
        7   & 34       & Robinson   & Elizabeth   & F      & 34       & Clerical    & 2           \\
        9   & 34       & Smith      & Jefferson   & M      & 34       & Clerical    & 2           \\
       11   & NULL     & Williams   & Serena      & NULL   & NULL     & NULL        & NULL        \\
      \end{tabulary}
    }
  \end{center}
 \end{frame}


\subsection{Right Outer Joins} %%%%%%%%%%%%%%%%%%%%%%%%%%%%%%%%%%%%%%%%%%%%%%%%%

\begin{frame}
  \frametitle{Right Outer Join: Discussion}
  TODO
\end{frame}

\begin{frame}
  \frametitle{Right Outer Join: Example}
  TODO
\end{frame}




\subsection{Full Outer Joins} %%%%%%%%%%%%%%%%%%%%%%%%%%%%%%%%%%%%%%%%%%%%%%%%%%

\begin{frame}
  \frametitle{Full Outer Join: Discussion}
  TODO
\end{frame}

\begin{frame}
  \frametitle{Full Outer Join: Example}
  TODO
\end{frame}


\section{Self Joins} %%%%%%%%%%%%%%%%%%%%%%%%%%%%%%%%%%%%%%%%%%%%%%%%%%%%%%%%%%%

\begin{frame}
  \frametitle{Self Join: Discussion}
  TODO
\end{frame}

\begin{frame}
  \frametitle{Self Join: Example}
  TODO
\end{frame}



\section{Questions?} %%%%%%%%%%%%%%%%%%%%%%%%%%%%%%%%%%%%%%%%%%%%%%%%%%%%%%%%%%%

\begin{frame}
  \frametitle{Additional Information}
  \begin{itemize}
  \item https://en.wikipedia.org/wiki/Join\_(SQL)
  \end{itemize}
\end{frame}

\begin{frame}
 \frametitle{Questions?}

\end{frame}

\end{document}
