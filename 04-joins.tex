\documentclass{beamer}
% 
% Choose how your presentation looks.
% 
% For more themes, color themes and font themes, see:
% http://deic.uab.es/~iblanes/beamer_gallery/index_by_theme.html
% 
\mode<presentation>
{
  \usetheme{Madrid}      % or try Darmstadt, Madrid, Warsaw, ...
  \usecolortheme{beaver} % or try albatross, beaver, crane, ...
  \usefonttheme{default}  % or try serif, structurebold, ...
  \setbeamertemplate{navigation symbols}{}
  \setbeamertemplate{caption}[numbered]
} 

\usepackage{graphicx}
\usepackage[english]{babel}
\usepackage[utf8x]{inputenc}
\usepackage{tabulary}

%% For Pretty In-Line Code:
\usepackage{listings}
\usepackage{color}

\definecolor{dkgreen}{rgb}{0,0.6,0}
\definecolor{gray}{rgb}{0.95,0.95,0.95}
\definecolor{mauve}{rgb}{0.58,0,0.82}

\lstdefinestyle{customsql}{
xleftmargin=4ex,
language=SQL
}


\lstset{frame=none,
  language=SQL,
  aboveskip=3mm,
  belowskip=3mm,
  showstringspaces=false,
  columns=flexible,
  basicstyle={\small\ttfamily},
  numbers=left,
  numberstyle=\tiny\color{black},
  keywordstyle=\color{blue},
  commentstyle=\color{dkgreen},
  backgroundcolor=\color{gray},
  stringstyle=\color{mauve},
  breaklines=true,
  breakatwhitespace=true,
  tabsize=2,
  captionpos=b,
  escapechar=@,
  style=customsql
}



\title[SQL Survival Guide]{SQL Survival Guide\\Joins}
%\author{Andy Choens, MSW}
\institute{Office of Quality and Patient Safety}
\date{May 18, 2015}

\begin{document}

\begin{frame}
  \titlepage
\end{frame}

\begin{frame}{Outline}
  \tableofcontents
\end{frame}



\section{What Is A Join Clause?} %%%%%%%%%%%%%%%%%%%%%%%%%%%%%%%%%%%%%%%%%%%%%%%

\begin{frame}
  \frametitle{What Is A Join Clause?}
  \pause
  \begin{itemize}
  \item Combines records from two or more tables.
  \item Returns a set of records which can be saved as a table or used
    as is.
  \item Joins are probably the hardest core SQL skill to master.
  \end{itemize}
\end{frame}  

\begin{frame}
  \frametitle{Why Do We Have To Do This?}
  \begin{itemize}
  \item Normalization: 
    \begin{itemize}
    \item There aren't enough Database Administrators to go around.
    \item Eliminates duplication of data.
    \item Makes it easier to maintain data integrity, at the cost of
      speed.
    \end{itemize}
  \item DOH DataMart is NOT consistently normalized:
    \begin{itemize}
    \item Normalization deserves its own presentation.
    \item Combines OLTP with Transactional Structures.
    \item Sometimes easier. Sometimes Harder
    \item Denormalized data == Mistakes.
    \item Use normalized data structure unless data is COMPLETELY static.
    \end{itemize}
  \end{itemize}
\end{frame}  



\section{Types of SQL Joins} %%%%%%%%%%%%%%%%%%%%%%%%%%%%%%%%%%%%%%%%%%%%%%%%%%%

\begin{frame}
  \frametitle{Types of Joins}

  There are 5 ANSI joins, plus one special case that
  deserves additional discussion:

  \begin{enumerate}
  \item Cross Join
  \item Inner Join
  \item Left Outer Join
  \item Right Outer Join
  \item Full Outer Join
  \item Self Join
  \end{enumerate}
\end{frame}  

\begin{frame}
  \frametitle{Data}

  \begin{itemize}
  \item We need example data.
  \item Two tables of data!
    \begin{itemize}
    \item Yes, you can join more than two tables at a time.
    \item No, we won't do that today.
    \end{itemize}
  \item Keep track of the next two slides.
  \item Code provided:
    \begin{itemize}
    \item sql: All SQL code needed to build tables and run
      queries. (Tested in Vertica.)
    \item data: Prepared sqlite db files. Can be used by any JDBC or
      ODBC compliant front-end.
    \end{itemize}
  \end{itemize}
 
\end{frame}

\begin{frame}
  \frametitle{Table: Departments}

  \begin{center}
    \begin{tabulary}{\textwidth}{CLC}
      DEPT\_ID & DEPT\_NAME  & DEPT\_FLOOR \\
      \hline
      31       & Sales       & 1           \\
      33       & Engineering & 3           \\
      34       & Clerical    & 2           \\
      35       & Marketing   & 3           \\
    \end{tabulary}
  \end{center}

  \bigskip
  Note: Aliased as 'empl'.

\end{frame}

\begin{frame}
  \frametitle{Table: Employees}

  \begin{center}
    \begin{tabulary}{\textwidth}{RCLLC}
      EID & DEPT\_ID & LAST\_NAME & FIRST\_NAME & GENDER\\
      \hline
      1   & 31       & Rafferty   & Gerry       & M     \\
      3   & 33       & Jones      & Jon         & M     \\
      5   & 33       & Heisenberg & Werner      & M     \\
      7   & 34       & Robinson   & Elizabeth   & F     \\
      9   & 34       & Smith      & Jefferson   & M     \\        
      11  & NULL     & Williams   & Serena      & F     \\
   \end{tabulary}
  \end{center}

  \bigskip
  Note: Aliased as 'empl'.
  
\end{frame}

\begin{frame}
  \frametitle{SQL Clauses}
  The following protected SQL commands START a clause:
  \smallskip
  \begin{itemize}
  \item SELECT
  \item FROM
  \item WHERE
  \item HAVING
  \end{itemize}

  \bigskip
  \pause
  Style:
  \smallskip
  \begin{itemize}
  \item Align SQL Clauses
  \item Predicates should be indented consistently.
  \item Feel free to ask me why I structure SQL queries the way I do.
  \end{itemize}
  
\end{frame}

\section{Examples} %%%%%%%%%%%%%%%%%%%%%%%%%%%%%%%%%%%%%%%%%%%%%%%%%%%%%%%%%%%%%

%% Cross Joins -----------------------------------------------------------------
\begin{frame}
  \frametitle{Cross Join: Discussion}
  \begin{itemize}
  \item Returns the Cartesian product of the tables in the FROM
    clause. 
  \item This is bad.
  \item These tables are tiny, so it doesn't really matter.
  \item But if you do this on a table like RECIP\_PROFILE, you are
    going to get a nasty call from Samir.
  \end{itemize}
\end{frame}

\begin{frame}[fragile]
  \frametitle{Explicit Cross Join: Example}

  Question: How many rows will the following query return?
  \bigskip

  \begin{lstlisting}[title={\tiny Source: https://github.com/Choens/sql-survival-guide/blob/master/sql/04-joins/cross-joins.sql}]
   SELECT *
   FROM EMPLOYEES empl cross join DEPARTMENTS dept
   ; 
  \end{lstlisting}
  
  \bigskip
  \pause 
  Answer: 24 rows!
  
\end{frame}

\begin{frame}
  \frametitle{Cross Join: Results}

  \begin{center}
    {\tiny
      \begin{tabulary}{\textwidth}{RCLLCRLC}
        EID & DEPT\_ID & LAST\_NAME & FIRST\_NAME & GENDER & DEPT\_ID & DEPT\_NAME  & DEPT\_FLOOR \\
        \hline
        1   & 31       & Rafferty   & Gerry       & M      & 31       & Sales       & 1           \\
        1   & 31       & Rafferty   & Gerry       & M      & 33       & Engineering & 3           \\
        1   & 31       & Rafferty   & Gerry       & M      & 34       & Clerical    & 2           \\
        1   & 31       & Rafferty   & Gerry       & M      & 35       & Marketing   & 3           \\
        3   & 33       & Jones      & Jon         & M      & 31       & Sales       & 1           \\
        3   & 33       & Jones      & Jon         & M      & 33       & Engineering & 3           \\
        3   & 33       & Jones      & Jon         & M      & 34       & Clerical    & 2           \\
        3   & 33       & Jones      & Jon         & M      & 35       & Marketing   & 3           \\
        5   & 33       & Heisenberg & Werner      & M      & 31       & Sales       & 1           \\
        5   & 33       & Heisenberg & Werner      & M      & 33       & Engineering & 3           \\
        5   & 33       & Heisenberg & Werner      & M      & 34       & Clerical    & 2           \\
        5   & 33       & Heisenberg & Werner      & M      & 35       & Marketing   & 3           \\
        7   & 34       & Robinson   & Elizabeth   & F      & 31       & Sales       & 1           \\
        7   & 34       & Robinson   & Elizabeth   & F      & 33       & Engineering & 3           \\
        7   & 34       & Robinson   & Elizabeth   & F      & 34       & Clerical    & 2           \\
        7   & 34       & Robinson   & Elizabeth   & F      & 35       & Marketing   & 3           \\
        9   & 34       & Smith      & Jefferson   & M      & 31       & Sales       & 1           \\
        9   & 34       & Smith      & Jefferson   & M      & 33       & Engineering & 3           \\
        9   & 34       & Smith      & Jefferson   & M      & 34       & Clerical    & 2           \\
        9   & 34       & Smith      & Jefferson   & M      & 35       & Marketing   & 3           \\
        11  & [NULL]   & Williams   & Serena      & F      & 31       & Sales       & 1           \\
        11  & [NULL]   & Williams   & Serena      & F      & 33       & Engineering & 3           \\
        11  & [NULL]   & Williams   & Serena      & F      & 34       & Clerical    & 2           \\
        11  & [NULL]   & Williams   & Serena      & F      & 35       & Marketing   & 3           \\
      \end{tabulary}
    }
  \end{center}

\end{frame}

\begin{frame}[containsverbatim]
  \frametitle{Implicit Cross Join: Example}
 
  \begin{lstlisting}[title={\tiny Source: https://github.com/Choens/sql-survival-guide/blob/master/sql/04-joins/cross-joins.sql}]
    SELECT *
    FROM EMPLOYEES empl, DEPARTMENTS dept
    ;
  \end{lstlisting}
  \bigskip
  \begin{itemize}
  \item Also returns 24 rows.
  \item You can do the same thing implicitly. (Don't!)
  \item What does this look like?
  \end{itemize}
  
\end{frame}

\begin{frame}
  \frametitle{Cross Join: Results}
  \begin{itemize}
  \item Cross Joins Are Dangerous!
  \item RECIP\_PROFILE == Phone Call From Samir.
  \item IMPLICIT CROSS JOIN Looks Like IMPLICIT INNER JOIN.
  \end{itemize}
\end{frame}

%% Inner Joins -----------------------------------------------------------------
\begin{frame}[containsverbatim]
  \frametitle{Inner Join: Discussion}
  \begin{lstlisting} 
    select * from EMPLOYEES;
  \end{lstlisting}
\end{frame}

\begin{frame}
  \frametitle{Inner Join: Example}
  
\end{frame}

%% Left Outer Joins ------------------------------------------------------------
\begin{frame}
  \frametitle{Left Outer Join: Discussion}
  TODO
\end{frame}

\begin{frame}
  \frametitle{Left Outer Join: Example}
  TODO
\end{frame}

%% Right Outer Joins -----------------------------------------------------------
\begin{frame}
  \frametitle{Right Outer Join: Discussion}
  TODO
\end{frame}

\begin{frame}
  \frametitle{Right Outer Join: Example}
  TODO
\end{frame}

%% Full Outer Joins ------------------------------------------------------------
\begin{frame}
  \frametitle{Full Outer Join: Discussion}
  TODO
\end{frame}

\begin{frame}
  \frametitle{Full Outer Join: Example}
  TODO
\end{frame}

%% Self Joins ------------------------------------------------------------------
\begin{frame}
  \frametitle{Self Join: Discussion}
  TODO
\end{frame}

\begin{frame}
  \frametitle{Self Join: Example}
  TODO
\end{frame}


\section{Questions?}

\begin{frame}
  \frametitle{Additional Information}
  \begin{itemize}
  \item https://en.wikipedia.org/wiki/Join\_(SQL)
  \end{itemize}
\end{frame}

\begin{frame}
 \frametitle{Questions?}

\end{frame}

\end{document}
