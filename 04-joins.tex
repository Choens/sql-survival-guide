\documentclass{beamer}
% 
% Choose how your presentation looks.
% 
% For more themes, color themes and font themes, see:
% http://deic.uab.es/~iblanes/beamer_gallery/index_by_theme.html
% 
\mode<presentation>
{
  \usetheme{Madrid}      % or try Darmstadt, Madrid, Warsaw, ...
  \usecolortheme{beaver} % or try albatross, beaver, crane, ...
  \usefonttheme{default}  % or try serif, structurebold, ...
  \setbeamertemplate{navigation symbols}{}
  \setbeamertemplate{caption}[numbered]
} 

\usepackage{graphicx}
\usepackage[english]{babel}
\usepackage[utf8x]{inputenc}
\usepackage{tabulary}

\title[SQL Survival Guide]{SQL Survival Guide\\Joins}
%\author{Andy Choens, MSW}
\institute{Office of Quality and Patient Safety}
\date{May 18, 2015}

\begin{document}

\begin{frame}
  \titlepage
\end{frame}

\begin{frame}{Outline}
  \tableofcontents
\end{frame}



\section{What Is A Join Clause?} %%%%%%%%%%%%%%%%%%%%%%%%%%%%%%%%%%%%%%%%%%%%%%%

\begin{frame}
  \frametitle{What Is A Join Clause?}
  \pause
  \begin{itemize}
  \item Combines records from two or more tables.
  \item Returns a set of records which can be saved as a table or used
    as is.
  \item Joins are probably the hardest core SQL skill to master.
  \end{itemize}
\end{frame}  

\begin{frame}
  \frametitle{Why Do We Have To Do This?}
  \begin{itemize}
  \item Normalization: 
    \begin{itemize}
    \item Eliminates duplication of data.
    \item Makes it easier to maintain data integrity, at the cost of
      speed.
    \end{itemize}
  \item DOH DataMart is NOT consistently normalized:
    \begin{itemize}
    \item Normalization deserves its own presentation.
    \item Combines OLTP with Transactional Structures.
    \item Sometimes easier. Sometimes Harder
    \item Denormalized data == Mistakes.
    \item Use normalized data structure unless data is COMPLETELY static.
    \end{itemize}
  \end{itemize}
\end{frame}  



\section{Types of SQL Joins} %%%%%%%%%%%%%%%%%%%%%%%%%%%%%%%%%%%%%%%%%%%%%%%%%%%

\begin{frame}
  \frametitle{Types of Joins}

  There are 5 ANSI joins, plus one special case that
  deserves additional discussion:

  \begin{enumerate}
  \item Inner Join
  \item Left Outer Join
  \item Right Outer Join
  \item Full Outer Join
  \item Cross
  \item Self Join
  \end{enumerate}
\end{frame}  

\begin{frame}
  \frametitle{Data}

  \begin{itemize}
  \item We need example data.
  \item The following two slides gives us two tables to use.
    \begin{itemize}
    \item Yes, you can join more than two tables at a time.
    \end{itemize}
  \item Keep track of these two table slides or you will get very
    lost.
  \item ALL SQL needed to produce the results in this presentation can
    be found in the data and sql folders, respectively.
  \end{itemize}
 
\end{frame}

\begin{frame}
  \frametitle{Offices}

  \begin{center}
    \begin{tabulary}{\textwidth}{CLC}
      DEPT\_ID & DEPT\_NAME  & DEPT\_FLOOR \\
      \hline
      31       & Sales       & 1           \\
      33       & Engineering & 3           \\
      34       & Clerical    & 2           \\
      35       & Marketing   & 3           \\
    \end{tabulary}
  \end{center}

\end{frame}

\begin{frame}
  \frametitle{Employees}

  \begin{center}
    \begin{tabulary}{\textwidth}{RCLLC}
      EID & DEPT\_ID & LAST\_NAME & FIRST\_NAME & GENDER\\
      \hline
      1   & 31       & Rafferty   & Gerry       & M     \\
      3   & 33       & Jones      & Jon         & M     \\
      5   & 33       & Heisenberg & Werner      & M     \\
      7   & 34       & Robinson   & Elizabeth   & F     \\
      9   & 34       & Smith      & Jefferson   & M     \\        
      11  & NULL     & Williams   & Serena      & F     \\
   \end{tabulary}
  \end{center}
  
\end{frame}

\section{Examples} %%%%%%%%%%%%%%%%%%%%%%%%%%%%%%%%%%%%%%%%%%%%%%%%%%%%%%%%%%%%%

%% Inner Joins -----------------------------------------------------------------
\begin{frame}
  \frametitle{Inner Join - Discussion}
  
\end{frame}

\begin{frame}
  \frametitle{Inner Join - Example}
  
\end{frame}

%% Left Outer Joins ------------------------------------------------------------
\begin{frame}
  \frametitle{Left Outer Join - Discussion}
  
\end{frame}

\begin{frame}
  \frametitle{Left Outer Join - Example}
  
\end{frame}

%% Right Outer Joins -----------------------------------------------------------
\begin{frame}
  \frametitle{Right Outer Join - Discussion}
  
\end{frame}

\begin{frame}
  \frametitle{Right Outer Join - Example}
  
\end{frame}

%% Full Outer Joins ------------------------------------------------------------
\begin{frame}
  \frametitle{Full Outer Join - Discussion}
  
\end{frame}

\begin{frame}
  \frametitle{Full Outer Join - Example}
  
\end{frame}

%% Cross Joins -----------------------------------------------------------------
\begin{frame}
  \frametitle{Cross Join - Discussion}
  
\end{frame}

\begin{frame}
  \frametitle{Cross Join - Example}
  
\end{frame}

%% Self Joins ------------------------------------------------------------------
\begin{frame}
  \frametitle{Self Join - Discussion}
  
\end{frame}

\begin{frame}
  \frametitle{Self Join - Example}
  
\end{frame}


\section{Questions?}

\begin{frame}
  \frametitle{Additional Information}
  \begin{itemize}
  \item https://en.wikipedia.org/wiki/Join\_(SQL)
  \end{itemize}
\end{frame}

\begin{frame}
 \frametitle{Questions?}

\end{frame}

\end{document}
